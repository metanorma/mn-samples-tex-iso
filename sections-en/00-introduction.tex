\section*{Introduction}

This document was developed in response to worldwide demand for minimum specifications for rice traded internationally, since most commercial bulks of grain, which have not been screened or aspirated, contain a proportion of other grains, weed seeds, chaff, straw, stones, sand, etc. The vegetable materials can have physical and biological properties which differ from those of the main constituent and can therefore affect the storage behaviour.

Rice is a permanent host to a considerable microflora; most of these microorganisms are cosmopolitan, the majority are innocuous, but some produce harmful by-products. Microflora communities present on freshly harvested rice include many types of bacteria, moulds and yeasts. While the rice is ripening and its moisture content is falling, the number of field microorganisms, mainly bacteria, diminishes. When the rice is harvested, it is invaded by storage microorganisms and the field microflora gradually dies out. If the mass fraction of moisture (formerly expressed as moisture content) is less than 18\%, the microflora does not multiply, whereas above 18\% it does so rapidly. Thus, at harvest, the qualitative and the quantitative composition of the microflora depends more upon ecological factors than upon the variety of the rice. During transport and storage, additions to the microfloral population occur. Microorganisms on the rice at harvest tend to die out during storage and are replaced by microorganisms adapted to storage conditions.

Storage losses have been estimated as being an average of 5\%, and as much as 30\%, especially in countries with climates favourable to the rapid development of agents of deterioration and where storage techniques are poorly developed, such as developing countries in the damp tropics. The magnitude of these figures highlights the need to promote throughout the world a rapid improvement in techniques of conservation.


\subsection*{Patent Notice}

The International Organization for Standardization (ISO) draws attention to the fact that it is claimed that compliance with this document may involve the use of a patent concerning sample dividers given in \ref{AnnexA} and shown in \ref{figureA-1}.

ISO takes no position concerning the evidence, validity and scope of this patent right.

The holder of this patent right has assured ISO that he/she is willing to negotiate licences under reasonable and non-discriminatory terms and conditions with applicants throughout the world. In this respect, the statement of the holder of this patent right is registered with ISO. Information may be obtained from:

\begin{flushleft}
  Vache Equipment\newline
  Fictitious\newline
  World\newline
  gehf@vacheequipment.fic
\end{flushleft}

Attention is drawn to the possibility that some of the elements of this document may be the subject of patent rights other than those identified above. ISO shall not be held responsible for identifying any or all such patent rights.
